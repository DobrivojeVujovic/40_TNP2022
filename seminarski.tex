\documentclass[a4paper]{article}

\usepackage{color}
\usepackage{url}
\usepackage[T2A]{fontenc} 
\usepackage[utf8]{inputenc} 
\usepackage{graphicx}

\usepackage[english,serbian]{babel}

\usepackage[unicode]{hyperref}
\hypersetup{colorlinks,citecolor=green,filecolor=green,linkcolor=blue,urlcolor=blue}

\begin{document}
	
	\title{Bežični prenos električne energije\\ \small{Seminarski rad u okviru kursa\\Tehničko i naučno pisanje\\ Matematički fakultet}}
	
	\author{Ime i prezime autora\\ kontakt email adresa autora}
	\date{/2022}
	\maketitle

\section{Primene}

Bežični prenos električne energije, ako pravilno implementiran, ima mogućnost da unapredi veliki broj industrija. Najveća potražnja je trenutno u sektoru potrošačke elektronike, gde jedna od prvih inovacija koje možemo da očekujemo jeste stanica za punjenje na distanci od 1-5 metara. \cite{1} Ideja je da u svakom domaćinstvu postoji par predajnika za punjenje (u zavisnosti od veličine prostora) koji bi punili svu kompatibilnu elektroniku i time potpuno uklonili potrebu za standardnim žičanim punjačima. Ovo rešenje je i dalje nemoguće, delom zbog nedovoljnog razvića tehnologije, ali i zbog skepticizma prosečnog potrošača o sigurnosti rešenja tog tipa. 
\\[10pt]
Kako je magnetna rezonanca u oblasti bežičnog prenosa energije proteklih decenija zadobila na popularnosti, isti principi magnetne teorije koji se koriste za tehnološki napredak elektronike mogu se primeniti i na medicinski sektor. \cite{2} BPE sistemi (eng. WPT systems) su na putu da preobraze proizvodnju medicinskih uređaja namenjenih za upotrebu unutar tela, specifično onih koji koriste baterije i električna kola. Pejsmejkeri (eng. pacemaker) su uređaji koji regulišu rad srca, a od kojih na globalnom nivou zavise životi oko 5 miliona ljudi. Vek trajanja baterije pejsmejkera (od 5-7 godina) \cite{3} pacijenta može dovesti u ozbiljan zdravstveni rizik, zbog čega naučnici rade na kreiranju uređaja koji se pune eksterno. 
\\[10pt]
Još neke od predviđenih primena:
... (lista)
\section{Održivost}
\label{sec:naslovM}

Standardne primene bežičnog prenosa energije nemaju dokazan štetan uticaj na životnu sredinu. U velikoj razmeri predviđa se da će BPE proizvoditi samo 20g ugljen-dioskida, jednog od glavnih uzroka globalnog zagrevanja, po kilovatsatu. \cite{1} Obračun za nuklearnu energiju je otprilike isti, dok ga nafta emitira čak i 40x više. *„Leed“ sertifikovani profesionalci smatraju da je tehnologija bezicnog prenosa energije sastavni deo pokreta održivosti i ’zelene’ gradnje \cite{4}, posebno kako maloprodajna cena elektriciteta nastavlja da raste. Bežicna energija pruza mogucnost smanjenja potrebe za novim ožičenjem kuća i preduzeća, kao i eliminisanja potencijalno rastuće potražnje za baterijama u budućnosti (posledica povećanja energetskih potreba IoT i 5G mreže).     

\bibliographystyle{unsrt}
\bibliography{references}

\end{document}
