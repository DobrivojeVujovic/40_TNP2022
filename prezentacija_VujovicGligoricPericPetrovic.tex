\documentclass[10pt]{beamer}
\usetheme{Madrid}
\usepackage[utf8]{inputenc}
\usepackage[english,serbian]{babel}
\usepackage{amsmath}
\usepackage{amsfonts}
\usepackage{amssymb}
\usepackage{graphicx}
\useinnertheme{circles}

\definecolor{UBCblue}{rgb}{0.03, 0.27, 0.49} % UBC Blue (primary)

\usecolortheme[named=UBCblue]{structure}

\def\d{{\fontencoding{T1}\selectfont\dj}}
\def\D{{\fontencoding{T1}\selectfont\DJ}}


\title{Bežični prenos električne energije}
\author{Dobrivoje Vujović, Mila Gligorić, Boris Perić, Milan Petrović}
\institute{Matematički fakultet Univerziteta u Beogradu}
\date{
	\footnotesize{25. decembar 2022.}	
}

\begin{document}
\begin{frame}
	\thispagestyle{empty}
	\titlepage
\end{frame}

\addtocounter{framenumber}{-1}

\begin{frame}{Literatura}
	\begin{itemize}
		
 	\item \textcolor{black}{{\footnotesize}}{\footnotesize \url{https://www.britannica.com/science/standing-wave-physics}}
	\item \textcolor{black}{{\footnotesize W. Bernard Carlson. Tesla: Inventor of the electrical age. Princeton University Press, 2013}}
	\item \textcolor{black}{{\footnotesize Emmanuel Chukwujioke: Wireless power transfer : The future. International Journal of Scientific and Engineering Research, 2015.}}
	\item \textcolor{black}{{\footnotesize A. David Wunsch:  Nikola tesla’s true wireless: A paradigm missed [scanning our past]. Proceedings of the IEEE, 106:1115–1123, 2018.}}
	\item \textcolor{black}{{\footnotesize John B. Kennedy. Collier magazine, 1926.}}
	\item \textcolor{black}{{\footnotesize}} {\footnotesize \url{https://siyss20.ungaforskare.se/Mengyang.li_report.pdf}}		
	\item \textcolor{black}{{\footnotesize Castellanos University of York Moore, Julian. Applications of wireless power transfer in medicine: State-of-the-art reviews. Journal of biomedical optics, 2019.}}
	\item \textcolor{black}{{\footnotesize}} {\footnotesize \url{https://blog.ossia.com/how-real-wireless-power-will-enable-a-more-sustainable-future}}		
	\end{itemize}
\end{frame}

\begin{frame}
	\frametitle{Sadržaj} % Table of contents slide, comment this block out to remove it
	\tableofcontents[hidesubsections] 
\end{frame}

\section{Uvod}

\begin{frame}\frametitle{Uvod}
	\begin{itemize}	
		\item Neefikasan prenos električne energije
		\item Visoka cena distribucije energije
		\item Krađa struje
	\end{itemize}
\end{frame}

\section{Istorija i pioniri}

\begin{frame}\frametitle{Istorija i pioniri}
	\begin{itemize}	
		\item Amper - veza struje i magneta
		\item Maksvelove jednačine
		\item Teslini pokušaji
		\item Braunova letelica na mikrotalasni pogon
		\item NASA-in laserski avion
	\end{itemize}
\end{frame}

\section{Princip rada}

\begin{frame}\frametitle{Princip rada}
	\begin{itemize}	
		\item ...
		\item ...
		\item ...
	\end{itemize}
\end{frame}

\section{Načini prenosa}

\begin{frame}\frametitle{Načini prenosa}
	\begin{itemize}
		\item Induktivni prenos
		\item Rezonantno induktivni prenos
		\item Mikrotalasi
		\item Laseri
\end{frame}

\section{Primene}

\begin{frame}\frametitle{Primene}
	Primene bežičnog prenosa energije su zaista mnogobrojne. Razvoj istog će znatno unaprediti naš svakodnevni život, a i promeniti način na koji pristupamo     određenim globalnim problemima. 
	\\Sfere u kojima će se bežični prenos koristiti:\\
	 \begin{itemize}
		\item potrošačka elektronika
		\item automobilska industrija
		\item biomedicina
		\item odbrambeni sektor itd.		 
	\end{itemize}
\end{frame}

\section{Zaključak}

\begin{frame}\frametitle{Zaključak}
	Ideja koja je pre nekoliko decenija bila samo san danas predstavlja glavni pokretač inovacija u bezbroj industrija.\\
	Najznacajnije ideje:\\
	\begin{itemize}
		\item Prenos energije mikrotalasnim putem i mogućnost da se time obezbedi neogranicena kolicina energije na Zemlji 
		\item Nove mogućnosti za istraživanje svemira
		\item Rešenje klimatske i energetske krize
		\item Potencijal da promeni planetu
	\end{itemize}
\end{frame}

\end{document}
