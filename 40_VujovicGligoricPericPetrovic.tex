\documentclass[a4paper]{article}

\usepackage{color}
\usepackage{url}
\usepackage[T2A]{fontenc} 
\usepackage[utf8]{inputenc} 
\usepackage{graphicx}

\usepackage[english,serbian]{babel}

\usepackage[unicode]{hyperref}
\usepackage{color}
\hypersetup{colorlinks,citecolor=green,filecolor=green,linkcolor=blue,urlcolor=blue}

\begin{document}

	\title{Bežični prenos električne energije\\ \small{Seminarski rad u okviru kursa\\Tehničko i naučno pisanje\\ Matematički fakultet}}
	
	\author{Dobrivoje Vujović \\mi22051@alas.matf.bg.ac.rs 
			\and Mila Gligorić \\mi22082@alas.matf.bg.ac.rs
			\and Milan Petrović \\mi22247@alas.matf.bg.ac.rs
			\and Boris Perić\\mi20185@alas.matf.bg.ac.rs}

	\date{15. Novembar. 2022.}
	\maketitle

\tableofcontents
\newpage

\section{Istorija i pioniri}

U 19. veku su se razvile mnoge teorije i kontrateorije o tome kako se električna energija može prenositi. Godine 1826, Andre-Mari Amper je otkrio vezu izmedju struje i magneta. Majkl Faradej je 1831. opisao svojim zakonom indukcije elektromotornu silu koja pokreće struju u petlji provodnika uz pomoć vremenski promenljivog magnetnog fluksa. Prenos električne energije bez žica su posmatrali mnogi pronalazači i eksperimentatori, ali nedostatak odgovarajuće teorije je nejasno pripisao ove pojave elektromagnetnoj indukciji.\cite{enwiki:1118144890}

Koncizno objašnjenje ovih fenomena dolazilo bi iz Maksvelovih jednačina iz 1860-ih Džejmsa Klerka Maksvela, uspostavljajući teoriju koja je ujedinila elektricitet i magnetizam sa elektromagnetizmom, predvidjajući postojanje elektromagnetnih talasa kao "bežičnog" nosioca elektromagnetne energije.
Nakon toga je usledila validacija teorije od strane Hajnriha Rudolfa Herca iz 1888. godine, koja je uključivala dokaze za radio talase.

Guljelmo Markoni je 1897. godine uspešno poslao Morseovu azbuku na rastojanje od 6 km.
Na slici 1 su prikazani inženjeri Britanske pošte kako pregledaju Markonijevu radio opremu tokom demonstracije na ostrvu Flat Holm, 13. maja 1897. Predajnik je u sredini, prijemnik ispod njega, a stub koji podržava žičanu antenu je vidljiv na vrhu.

\begin{figure}[h!]
    \centering
    \includegraphics[scale=0.2]{Markoni.jpg}
    \caption{Markoni i inženjeri}
    \label{fig:my_labela}
\end{figure}



U medjuvremenu Tesla počinje ozbiljno proučavati ideju bežičnog prenosa električne energije. Njegova zamisao je bila da bežičnim putem svetu obezbedi besplatnu elekričnu energiju. 1893. osvetljava sijalicu bežičnim putem, a nedugo zatim započinje gradnju Vardenklif tornja
koji bi se koristio za prenos električne energije bežičnim putem na velike udaljenosti. Zbog velikih gubitaka projekt je ostao bez finansijskih sredstava i na kraju je ugašen. 

Vilijam Braun je 1964. pokazao model vazduhoplova na mikrotalasni pogon koji je svu
za let potrebnu energiju dobivao iz mikrotalasnih zraka.\cite{brown1996history}
2003. NASA je letela prvim laserskim avionom. Motor malog modela aviona pokretala je električna energija koju su generisale fotoćelije iz snopa infracrvene svetlosti iz zemaljskog lasera, dok je kontrolni sistem držao laser uperen u avion.

\begin{figure}[h!]
    \centering
    \includegraphics[scale=0.4]{aircraft.jpg}
    \caption{Prvi let NASA lasersog aviona}
    \label{fig:my_label}
\end{figure}


2008. Intel je pokazao da se može bežično poslati energija da bi se upalila sijalica sa efikasnošću prenosa 75 upešnih pokušaja na 100 pokušaja.



\section{Primene}
\label{sec:Primene}
Bežični prenos električne energije, ako pravilno implementiran, ima mogućnost da unapredi veliki broj industrija. Najveća potražnja je trenutno u sektoru potrošačke elektronike, gde jedna od prvih inovacija koje možemo da očekujemo jeste stanica za punjenje na distanci od 1-5 metara. \cite{a} Ideja je da u svakom domaćinstvu postoji par predajnika za punjenje (u zavisnosti od veličine prostora) koji bi punili svu kompatibilnu elektroniku i time potpuno uklonili potrebu za standardnim žičanim punjačima. Ovo rešenje je i dalje nemoguće, delom zbog nedovoljnog razvića tehnologije, ali i zbog skepticizma prosečnog potrošača o sigurnosti rešenja tog tipa. 
\\[10pt]
Kako je magnetna rezonanca u oblasti bežičnog prenosa energije proteklih decenija zadobila na popularnosti, isti principi magnetne teorije koji se koriste za tehnološki napredak elektronike mogu se primeniti i na medicinski sektor. \cite{b} BPE sistemi (eng. WPT systems) su na putu da preobraze proizvodnju medicinskih uređaja namenjenih za upotrebu unutar tela, specifično onih koji koriste baterije i električna kola. Pejsmejkeri (eng. pacemaker) su uređaji koji regulišu rad srca, a od kojih na globalnom nivou zavise životi oko 5 miliona ljudi. Vek trajanja baterije pejsmejkera (od 5-7 godina) \cite{c} pacijenta može dovesti u ozbiljan zdravstveni rizik, zbog čega naučnici rade na kreiranju uređaja koji se pune eksterno. 
\\[10pt]
Bežični prenos energije je na putu da unapredi i automobilsku industriju. Kako se svake godine proizvodi sve više električnih vozila, traže se i inovativni načini za njihovo punjenje. Trenutno se radi na stanicama gde automobili mogu parkirani da se pune bežično. U planu za budućnost je izrada tehnologije pomoću koje će automobili moći da se napune čak i dok su u pokretu, a punjaći će biti raspoređeni po putevima tako da će vozila moći da rade bez stajanja. Kao rezultat ove promene, mnogo više ljudi bi biralo električna vozila. 
\\[10pt]
Još neke sfere u kojima koje će se bežični prenos energije koristiti:
\begin{itemize}
 \item gradjevina: uklanjanje žica iz domova
 \item odbrambeni sektor: smanjenje težine baterija na bojnom polju
 \item biomedicina: izrada protetskih uredjaja
 \item solarne ploče itd.
\end{itemize}
\section{Održivost}
\label{sec:Održivost}

Standardne primene bežičnog prenosa energije nemaju dokazan štetan uticaj na životnu sredinu. U velikoj razmeri predviđa se da će BPE proizvoditi samo 20g ugljen-dioskida, jednog od glavnih uzroka globalnog zagrevanja, po kilovatsatu. \cite{a} Obračun za nuklearnu energiju je otprilike isti, dok ga nafta emitira čak i 40x više. *„Leed“ sertifikovani profesionalci smatraju da je tehnologija bezicnog prenosa energije sastavni deo pokreta održivosti i ’zelene’ gradnje \cite{d}, posebno kako maloprodajna cena elektriciteta nastavlja da raste. Bežicna energija pruza mogucnost smanjenja potrebe za novim ožičenjem kuća i preduzeća, kao i eliminisanja potencijalno rastuće potražnje za baterijama u budućnosti (posledica povećanja energetskih potreba IoT i 5G mreže).     


\section{Svetski bežicni sistem}
\label{sec:svetskisistem}
Svetski bežicni sistem je bio projekat koji je dizajnirao Nikola Tesla na osnovu svojih teorija o korišćenju Zemlje I njene atmosfere kao električnih provodnika. Bitnost ovog projekta ogleda se u činjenici da je njime Tesla pokazao da je moguce prenositi energiju na daljinu, bez korišćenja drugih provodnika, osim vazduha. On je tvrdio da će ovaj sistem omogućiti “Prenos električne energije bez žica” na globalnom nivou, kao I bežicnu komunikaciju I emitovanje radio signala. Projekat je napusten 1906. godine zbog nedostatka sredstava I nikada posle toga nije završen.

\begin{figure}[h!]
\begin{center}
\includegraphics[scale=0.25]{toranj.jpg}
\end{center}
\caption{Teslin Toranj}
\label{fig:toranj}
\end{figure}

\subsection{Istorija}
\label{subsec:istorija}
Kada se Tesla vratio iz Kolorado Springsa u Njujork, napisao je senzacionalan članak za Century Magazine\cite{teslinClanak}. U ovoj detaljnoj, futurističkoj viziji on je opisao sredstvo za prikupljanje sunčeve energije pomoću antene. On je sugerisao da bi bilo moguce kontrolisati vremenske prilike električnom energijom. Predvideo je mašine koje će rat učiniti nemogućim. I predložio je globalni sistem bežične komunikacije. Većini ljudi ideje su bile gotovo nerazumljive, ali Tesla je bio čovek koji se nije mogao potceniti.

Ovaj članak privukao je pažnju jednog od najmoćnijih ljudi toga vremena, J.P Morgan-a. Budući da su se lično poznavali, Tesla je Morganu izložio svoju ideju, koja se tada graničila sa naučnom fantastikom. “Kada se bežična mreža u potpunosti primeni, Zemlja će se pretvoriti u ogroman mozak, sposoban da reaguje u svakom svome delu”\cite{intervju1926}, tvrdio je Tesla.

Morgan je Tesli ponudio 150 000 dolara, da napravi prenosni toranj I elektranu. Iako je Tesli trebalo mnogo više novca za projekat, prihvatio je ponudu I odmah krenuo sa radovima.


Kao mesto za svoj projekat Tesla je izabrao zemljište Vardenklif(engl. Wardenclyffe) na liticama Long Island-a. Projekat je bio u izgradnji sve do 1901. godine, a konstrukcija je uključivala toranj koji se uzdizao 57m u visinu I koji je na svom vrhu nosio čeličnu kuglu tešku 55 tona. Ispod tornja, 35 metara duboko, nalazilo se udubljenje, nalik na bunar, a par metara dublje nalazilo je se šesnaest gvozdenih cevi učvršćenih u zemlju.

Kako je izgradnja tornja odmicala, postalo je jasno da početna investicija neće biti dovoljna, ali Morgan nije reagovao. Zatim je krajem 1901. objavljena vest da je Markoni(ital. Guglielmo Marconi) signalizirao slovo “S” preko Atlantika od Kornvola u Engleskoj do Njufaundlenda, što je dalje poljuljalo Markonija u odluci da nastavi da finansira Teslin projekat.

Tesla je molio Morgana za dodatnu finansijsku podršku, ali je investitor odlučno odbio; berza je doživela krah, a cene materijala su se udvostručile. Visoke cene u kombinaciji sa neuspelim pokušajem pronalaska investitora, na kraju su dovele do propasti projekta.

\subsection{Princip Rada}
\label{subsec:principrada}
Teslina ideja za Vardenklifski toranj rodila se početkom 1890-ih godina. Glavni cilj njegovog istraživanja bio je da razvije novi sistem bežičnog prenosa energije. Odbacio je ideju o korišćenju novootkrivenih Hercovih radio Talasa, koje je 1888. otkrio nemački fizičar Hajnrih Rudolf Herc. Tesla je sumnjao u njihovo postojanje, pošto mu je osnovna fizika, kao I većini drugih naučnika iz tog perioda, govorila da će talasi  putovati pravolinijski, što znači da će putovati pravo u svemir, postajući izgubljeni.\cite{izgubljeniTalasi}

Tokom prethodnih eksperimenata izvedenih u Kolorado Springsu 1899. godine, Tesla je razvio sopstvene ideje o tome kako će funkcionisati svetski bežični sistem. Iz ovih eksperimenata izneo je teoriju da bi ako bi pustio električnu struju u Zemlju na određenoj frekvenciji, mogao da iskoristi ono za sta je verovao da je Zemljin sopstveni električni potencijal, I da izazove da struja rezonuje na frekvenciji koja bi se pojačavala po principu stojecih talasa\cite{stojeciTalasi}. Energiji ovih talasa moglo bi se pristupiti bilo gde na Zemlji za pokretanje raznih uređaja. Ovakav sistem zasnivao je se više na idejama o električnoj provodljivosti iz 19. veka umesto o modernijim teorijama o elektromagnetnim talasima.
	
Februara 1901. u Collier-ovom članku pod nazivom “Razgovor sa planetama Tesla je opisao svoj bežični sistem na sledeći način:
“Koristi Zemlju kao medijum za sprovođenje struje, oslobađajući se žica I svih drugih veštačkih provodnika ...mašina koja, da objasnim svoj rad jednostavnim jezikom, u svom delovanju podseća na pumpu, crpeći struju Iz Zemlje I vraćajući je nazad u ogromnoj brzini, stvarajući tako talasanje ili potrerse koji bi, šireći se Zemljom kao kroz žicu, mogli biti otkriveni na velikoj udaljenosti pažljivo podešenim prijemnim krugovima. Na ovaj način mogao sam da prenesem na daljinu, ne samo slabe signale, već i značajne količine energije, a kasnija otkrića koje sam imao, uverila su me da ću na kraju uspeti da prenosim energiju bežično u industrijske svrhe, uz niske cene I na bilo kojoj udaljenosti.

Iako je Tesla demonstrirao bežični prenos energije osvetljavajući sijalice postavljene ispred zgrade u kojoj je postavio svoj eksperimentalni kalem,\cite{teslinClanak} nije naučno dokazao svoje teorije. Verovao je da je postigao rezonaciju Zemlje koja bi, prema njegovoj teoriji, delovala na bilo kojoj udaljenosti.\cite{teorija}


\section{Načini prenosa energije}

Za prenos eklektrične energije koriste se tehnologije za prenos na blizinu i na velike razdaljine odnosno near-field (neradijativna) i far-field(radijativna) tehnologija.



\begin{table}[h]
    \centering
    \caption{Usporedba tehnologija bežičnog prenosa energije}
    
    \begin{tabular}{|p{2 cm}|p{2 cm}|p{2 cm}|p{2 cm}|p{2 cm}|}
        \hline
    & induktivno  povezivanje & rezonantno induktivno povezivanje & mikrovalni prenos & laserski prenos \\ \hline
    primljena snaga  & nekoliko W & stotine W & do desetine kW & očekivano do MW\\ \hline
    moguća udaljenost & nekoliko cm & nekoliko m & do desetine km & do desetine km \\ \hline
    frekvencija rada & nekoliko kHz do desetine kHz & nekoliko MHz & do 300 GHz & više od 1 THz \\ \hline
    efikasnost & Vrlo visoka & visoka & srednja & srednja \\ \hline
        
    
    \end{tabular}
    \end{table}
    
    


\subsection{Prenos energije na male udaljenosti}
\label{subsec:prenosenergijenamaleudaljenosti}


Near-field se zasniva na principu da se u polju dometa električni i magnetni talasi razdvajaju.
Ovaj princip se zasniva na elektromagnetnoj indukciji između žica i namotaja pomoću koje se energija prenosi putem magnetnog polja odnosno elektrostatičke indukcije između dve metalne elektrode kod električnog polja.
Kod ovakog prenosa odašiljač i prijemnik se mogu povezati indukcijom i rezonantnom indukcijom.
Kod obične indukcije udaljenost je svega par centimetara dok je kod rezonantne indukcije prenos moguć i na nekoliko metara. Jedna od glavnih odlika ovog pristupa je izuzetno velika efikasnost.

Induktivno povezivanje zahteva svega nekoliko W struje pa je zbog svog malog dometa potrebna i mala frekvencija talasa od svega nekoliko kHz i što je čini najefikasnijom metodom za prenos energije, njena primena je u glavnom punjenje i napajanje malih uređaja u domaćinstvu.

Rezonantno induktivno povezivanje koristi više energije od induktivnog s toga mu je i domet veći i pa je i frekvencija u opsegu od nekoliko MHz, efikasnost ovog prenosa je malo manja od običnog induktivnog povezivanja ali je i dalje visoka.

\subsection{Prenos energije na velike udaljenosti}
\label{subsec:prenosenergijenavelikeudaljenosti}


Far-field tehnologije služe za prenos na mnogo veće udaljenosti i za takav prenos energije se koriste mikro talasi odnosno laseri.

Mikrotalasi koriste par hiljada puta više energije od indukcije odnosno do par desetina kW, njihov domet zajedno sa laserima prelazi desetine kilometara i opseg frekvencija ovih talasa se kreće od 300MHz do 300GHz. Sistem se sastoji od generatora talasa koji naizmeničnu struju pretvara u jednosmernu a nakon toga jednosmernu struju u elektromagnetne talase. Pomoću antene za prijem, koja se drugačije nazivaju i rectenna, elektromagnetni talasi se pretvaraju u jednosmernu struju. Ideja iza ovog principa je ta da se u orbitu lansiraju svemirske stanice koje bi pomoću solarnih panela proizvodile struju i emitovale je nazad na Zemlju.

Problem kod ovog načina slanja energije je sto je aparatura neophodna za njen rad ogromna i antena za slanje zauzima 1km u prečniku dok antena za prijem zauzima preko 10km prečnika. Sa druge strane plantaža solarnih panela ovih dimenzija može da proizvede 10 hiljada MW elektricne energije u poređenju sa 750 MW koje bi antena mogla da primi.

Takođe jedna od mana ovog pristupa je lomljenje talas u susretu sa preprekama što rezultuje gubitku efikasnosti.

Laseri su gusto zbijeni svetosni talasi koji za razliku od elektromagnetnih mogu da budu vidljivi. Princip rada lasera je da se velike kolicine struje pretvore u laserski zrak i direktno upere u prijemnik koji će laser pomoću fotonaponskih ćelija pretvoriti u jednosmernu struju.

Prednost lasera je ta što su talasi monohromatski i jednakih talasnih duzina što omogucava optimizaciju u nalaženju idealnog talasa koji ima najbolju stopu prevođenja svetlosti u električnu energiju. Takođe aparatura je znatno manja i kompaktnija i laserski zraci ne ometaju radio komunikacije za razliku od elektromagnetnih.

Sa druge strane mane ovog pristupa su te što su laseri štetni po ljude i zahtevaju velike sigurnosne protokole za ispravan rad, u poređenju sa prijemnicima koji su znatno manji nego kod mikrotalasa, hladnjaci za fotoćelije su ogromni zbog količine energije koja se nalazi u laseru i trenutno maloj efikasnosti pretvaranja svetlosti u fotonaponskim ćelijama. Atmosferski uslovi kao magla i kiša u potpunosti mogu blokirati lasere što ih čini beskorisnim u lošim vremenskim uslovima jer za funkcionisanje potrebno je laser uperiti direktno u prijemnik sto je još jedan problem ako između odasiljača i prijemnika ne postoji direktna linija već se između nalaze prepreke. Ovo se može izbeći korišćenjem optičkih kablova čija efikasnost takođe nije optimalna za funkcionisanje.

\bibliographystyle{plain}
\bibliography{seminarski.bib}

\end{document}
